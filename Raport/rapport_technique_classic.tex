\documentclass[12pt,a4paper,twoside]{report}

\usepackage[utf8]{inputenc}
\usepackage[french]{babel}
\usepackage[T1]{fontenc}
\usepackage{geometry}
\usepackage{graphicx}
\usepackage{amsmath,amssymb}
\usepackage{listings}
\usepackage{xcolor}
\usepackage{hyperref}
\usepackage{booktabs}
\usepackage{longtable}
\usepackage{fancyhdr}
\usepackage{titlesec}
\usepackage{tcolorbox}
\usepackage{enumitem}
\usepackage{multirow,array}
\usepackage{caption,subcaption}
\usepackage{tikz}
\usepackage{setspace}

\geometry{a4paper,left=3cm,right=2.5cm,top=2.5cm,bottom=2.5cm,headheight=15pt}

\definecolor{primaryblue}{RGB}{31,119,180}
\definecolor{secondaryblue}{RGB}{174,199,232}
\definecolor{darkgray}{RGB}{51,51,51}
\definecolor{lightgray}{RGB}{240,240,240}

\lstset{
    backgroundcolor=\color{lightgray},
    commentstyle=\color{green!60!black}\itshape,
    keywordstyle=\color{primaryblue}\bfseries,
    numberstyle=\tiny\color{darkgray},
    stringstyle=\color{purple},
    basicstyle=\ttfamily\small,
    breaklines=true,
    numbers=left,
    frame=single,
    captionpos=b
}

\hypersetup{
    colorlinks=true,
    linkcolor=primaryblue,
    urlcolor=primaryblue,
    pdftitle={Rapport Technique - Plateforme d'Optimisation des EDT},
    bookmarksopen=true
}

\pagestyle{fancy}
\fancyhf{}
\fancyhead[LE]{\slshape\nouppercase{\leftmark}}
\fancyhead[RO]{\slshape\nouppercase{\rightmark}}
\fancyfoot[C]{\thepage}
\renewcommand{\headrulewidth}{0.5pt}

\titleformat{\chapter}[display]
  {\normalfont\huge\bfseries\color{primaryblue}}
  {\chaptertitlename\ \thechapter}{20pt}{\Huge}

\setlength{\parindent}{0pt}
\setlength{\parskip}{6pt}
\onehalfspacing

\begin{document}

\begin{titlepage}
    \centering
    \vspace*{1cm}
    
    {\color{primaryblue}\rule{\textwidth}{2pt}}
    \vspace{0.5cm}
    
    {\Huge\bfseries\color{primaryblue} PLATEFORME D'OPTIMISATION\\[0.3cm]
    DES EMPLOIS DU TEMPS\\[0.3cm]
    D'EXAMENS UNIVERSITAIRES\\[1cm]}
    
    {\color{primaryblue}\rule{\textwidth}{2pt}}
    \vspace{1.5cm}
    
    {\LARGE\bfseries Rapport Technique Complet\\[0.5cm]}
    
    \vspace{1cm}
    
    \begin{tcolorbox}[colback=lightgray,colframe=primaryblue,width=0.8\textwidth,arc=3mm,boxrule=1.5pt]
        \centering
        {\Large\bfseries Système de Gestion Intelligent}\\[0.3cm]
        {\large PostgreSQL • Python • Streamlit}\\[0.3cm]
        {\large 13,051 Étudiants • 110 Formations • 1,118 Modules}
    \end{tcolorbox}
    
    \vspace{2cm}
    
    {\Large\bfseries Projet de Base de Données Avancées\\[0.3cm]}
    {\large Année Universitaire 2024-2025\\[1cm]}
    
    \vfill
    
    \begin{tabular}{rl}
        \textbf{Auteur:} & Projet Académique \\[0.2cm]
        \textbf{Encadrant:} & Département Informatique \\[0.2cm]
        \textbf{Date:} & \today \\[0.2cm]
        \textbf{Version:} & 2.0 - Classic Edition
    \end{tabular}
    
    \vspace{1cm}
    {\color{primaryblue}\rule{\textwidth}{2pt}}
\end{titlepage}

\cleardoublepage
\pagenumbering{roman}

\chapter*{Résumé Exécutif}
\addcontentsline{toc}{chapter}{Résumé Exécutif}

Ce rapport présente la conception, l'implémentation et l'évaluation d'une plateforme complète d'optimisation des emplois du temps d'examens universitaires développée pour gérer plus de 13,000 étudiants, 110 formations et 1,118 modules à travers 7 départements.

\section*{Objectifs et Portée}

Le système vise à automatiser et optimiser la génération des emplois du temps d'examens tout en garantissant le respect de contraintes multiples et complexes. Les résultats démontrent une réussite complète avec 100\% de tests validés et une génération d'emploi du temps en moins de 2 minutes.

\section*{Architecture Technique}

\begin{itemize}[leftmargin=*]
    \item \textbf{Base de données}: PostgreSQL 14.18 avec 10 tables, 8 vues analytiques
    \item \textbf{Backend}: Python 3.9+ avec algorithmes optimisés
    \item \textbf{Frontend}: Streamlit 1.29.0 avec 5 interfaces utilisateur
    \item \textbf{Performance}: Génération en 78 secondes, 0 conflit critique
\end{itemize}

\section*{Résultats Clés}

\begin{table}[h]
\centering
\begin{tabular}{@{}lr@{}}
\toprule
\textbf{Métrique} & \textbf{Valeur} \\ \midrule
Taux de réussite des tests & 100\% (9/9) \\
Examens planifiés & 1,118 \\
Temps de génération & 78 secondes \\
Conflits détectés & 0 critiques \\
Intégrité des données & 100\% validée \\ \bottomrule
\end{tabular}
\caption{Résultats de performance du système}
\end{table}

\cleardoublepage
\tableofcontents
\cleardoublepage

\pagenumbering{arabic}

\chapter{Introduction}

\section{Contexte et Problématique}

La gestion des emplois du temps d'examens dans une université moderne représente un défi organisationnel majeur nécessitant la coordination de multiples ressources tout en respectant des contraintes strictes.

\subsection{Défis de la Planification Manuelle}

\begin{itemize}[leftmargin=*]
    \item Temps de préparation: 2-3 semaines pour un planning complet
    \item Risque élevé d'erreurs humaines et de conflits
    \item Sous-utilisation des ressources disponibles
    \item Difficulté à gérer les modifications de dernière minute
\end{itemize}

\subsection{Solution Proposée}

Ce projet développe un système informatique capable de générer automatiquement des emplois du temps optimisés en respectant l'ensemble des contraintes réglementaires et pédagogiques.

\section{Objectifs du Projet}

\begin{table}[h]
\centering
\begin{tabular}{@{}p{0.4\textwidth}p{0.5\textwidth}@{}}
\toprule
\textbf{Objectif} & \textbf{Critère de Succès} \\ \midrule
Génération automatique & Planning complet en < 2 minutes \\
Respect des contraintes & 100\% des contraintes critiques \\
Détection de conflits & Identification automatique \\
Interfaces utilisateur & 5 interfaces adaptées \\
Performance & Support de 20,000+ étudiants \\ \bottomrule
\end{tabular}
\caption{Objectifs et critères de succès}
\end{table}

\section{Périmètre}

Le système couvre 7 départements, 110 formations, 13,051 étudiants, 148 professeurs, 1,118 modules et 126 salles.

\chapter{Architecture Technique}

\section{Vue d'Ensemble}

Le système adopte une architecture en trois couches distinctes:

\begin{enumerate}[leftmargin=*]
    \item \textbf{Couche Données}: PostgreSQL avec tables, vues et fonctions
    \item \textbf{Couche Métier}: Python avec modules spécialisés
    \item \textbf{Couche Présentation}: Streamlit avec interfaces interactives
\end{enumerate}

\section{Stack Technologique}

\begin{table}[h]
\centering
\begin{tabular}{@{}ll@{}}
\toprule
\textbf{Composant} & \textbf{Technologie} \\ \midrule
SGBD & PostgreSQL 14.18 \\
Backend & Python 3.9+ \\
Framework Web & Streamlit 1.29.0 \\
Connecteur BD & psycopg2-binary 2.9.9 \\
Visualisation & Plotly 5.18.0 \\
Analyse & Pandas 2.1.4, NumPy 1.26.2 \\
Export Excel & openpyxl 3.1.5 \\ \bottomrule
\end{tabular}
\caption{Stack technologique}
\end{table}

\section{Modules Python}

\begin{itemize}[leftmargin=*]
    \item \texttt{database.py}: Gestion des connexions et requêtes
    \item \texttt{constraints.py}: Vérification des contraintes
    \item \texttt{fast\_scheduler.py}: Algorithme optimisé de planification
    \item \texttt{analytics.py}: Calcul des KPIs et statistiques
\end{itemize}

\chapter{Modèle de Données}

\section{Schéma Relationnel}

La base de données comprend 10 tables principales:

\begin{table}[h]
\centering
\small
\begin{tabular}{@{}lll@{}}
\toprule
\textbf{Table} & \textbf{Rôle} & \textbf{Cardinalité} \\ \midrule
departements & Départements universitaires & 7 \\
formations & Programmes d'études & 110 \\
etudiants & Étudiants inscrits & 13,051 \\
professeurs & Corps enseignant & 148 \\
modules & Modules d'enseignement & 1,118 \\
inscriptions & Inscriptions & 105,468 \\
lieu\_examen & Salles et amphithéâtres & 126 \\
periodes\_examen & Périodes d'examens & 1 \\
examens & Examens planifiés & 1,118 \\
surveillances & Affectations & Variable \\ \bottomrule
\end{tabular}
\caption{Tables principales}
\end{table}

\section{Vues Analytiques}

Huit vues fournissent des analyses en temps réel:

\begin{enumerate}[leftmargin=*]
    \item \texttt{kpi\_global}: Indicateurs clés globaux
    \item \texttt{stats\_departement}: Statistiques par département
    \item \texttt{conflits\_etudiants}: Détection conflits étudiants
    \item \texttt{conflits\_professeurs}: Détection conflits professeurs
    \item \texttt{conflits\_capacite}: Dépassements de capacité
    \item \texttt{conflits\_salles}: Conflits d'occupation
    \item \texttt{occupation\_salles}: Taux d'occupation
    \item \texttt{charge\_professeurs}: Charge de travail
\end{enumerate}

\chapter{Algorithme de Planification}

\section{Principe FastScheduler}

L'algorithme FastScheduler utilise une approche heuristique optimisée:

\begin{enumerate}[leftmargin=*]
    \item Tri des modules par nombre d'inscrits (décroissant)
    \item Allocation séquentielle avec vérification en mémoire
    \item Insertion par lots pour performance maximale
    \item Optimisation post-génération
\end{enumerate}

\section{Contraintes Respectées}

\begin{itemize}[leftmargin=*]
    \item Maximum 1 examen par jour par étudiant
    \item Maximum 3 examens par jour par professeur
    \item Capacité des salles respectée
    \item Pas de chevauchement de salles
    \item Créneaux: 8h30, 11h00, 14h30, 17h00
\end{itemize}

\section{Performance}

\begin{table}[h]
\centering
\begin{tabular}{@{}lr@{}}
\toprule
\textbf{Métrique} & \textbf{Valeur} \\ \midrule
Temps de génération & 78 secondes \\
Modules traités & 1,118 \\
Examens planifiés & 1,118 (100\%) \\
Échecs & 0 \\
Conflits critiques & 0 \\ \bottomrule
\end{tabular}
\caption{Performance de l'algorithme}
\end{table}

\chapter{Interfaces Utilisateur}

\section{Pages Streamlit}

Le système offre 5 interfaces adaptées:

\begin{table}[h]
\centering
\begin{tabular}{@{}lp{0.6\textwidth}@{}}
\toprule
\textbf{Page} & \textbf{Fonctionnalités} \\ \midrule
Dashboard & KPIs globaux, détection de conflits \\
Administration & Génération EDT, optimisation \\
Statistiques & Analyses stratégiques, graphiques \\
Départements & Vues départementales \\
Consultation & Plannings personnalisés \\ \bottomrule
\end{tabular}
\caption{Pages de l'application}
\end{table}

\section{Fonctionnalités Clés}

\begin{itemize}[leftmargin=*]
    \item Génération automatique en un clic
    \item Visualisations interactives avec Plotly
    \item Export CSV et Excel
    \item Détection automatique de conflits
    \item Planning par formation (nouveau)
\end{itemize}

\chapter{Tests et Validation}

\section{Résultats des Tests}

\begin{table}[h]
\centering
\begin{tabular}{@{}lcc@{}}
\toprule
\textbf{Catégorie} & \textbf{Statut} & \textbf{Détails} \\ \midrule
Connexion BD & ✓ & PostgreSQL 14.18 \\
Tables & ✓ & 10/10 \\
Vues & ✓ & 8/8 \\
Fonctions & ✓ & 2/2 \\
Méthodes Python & ✓ & 10/10 \\
Contraintes & ✓ & Tous types détectés \\
Analytique & ✓ & 5/5 fonctions \\
Planification & ✓ & 1,118 examens \\
Intégrité & ✓ & 0 orphelins \\ \bottomrule
\end{tabular}
\caption{Résultats des tests - 100\% réussite}
\end{table}

\section{Performance des Requêtes}

\begin{table}[h]
\centering
\begin{tabular}{@{}lrr@{}}
\toprule
\textbf{Requête} & \textbf{Temps (ms)} & \textbf{Résultats} \\ \midrule
KPIs Globaux & 54.93 & 1 \\
Stats Départements & 2,864.90 & 7 \\
Conflits Étudiants & 14.65 & 0 \\
Conflits Professeurs & 13.03 & 0 \\
Liste Modules & 27.45 & 1,118 \\ \bottomrule
\end{tabular}
\caption{Performance des requêtes SQL}
\end{table}

\chapter{Déploiement}

\section{Options de Déploiement}

\begin{table}[h]
\centering
\begin{tabular}{@{}lccc@{}}
\toprule
\textbf{Plateforme} & \textbf{Coût} & \textbf{Difficulté} & \textbf{Recommandé} \\ \midrule
Streamlit Cloud & Gratuit & Facile & Démo \\
Railway & \$5/mois & Moyen & Production \\
Serveur Universitaire & Gratuit & Moyen & Optimal \\ \bottomrule
\end{tabular}
\caption{Options de déploiement}
\end{table}

\section{Configuration Requise}

\begin{itemize}[leftmargin=*]
    \item Serveur: 4GB RAM, 2 CPU cores
    \item PostgreSQL 14+
    \item Python 3.9+
    \item Connexion internet (pour Streamlit Cloud)
\end{itemize}

\chapter{Conclusion}

\section{Objectifs Atteints}

Le projet a réussi à:

\begin{itemize}[leftmargin=*]
    \item ✓ Implémenter un système complet de gestion d'examens
    \item ✓ Gérer plus de 100,000 inscriptions
    \item ✓ Respecter toutes les contraintes critiques
    \item ✓ Fournir des interfaces utilisateur intuitives
    \item ✓ Garantir l'intégrité des données (100\%)
    \item ✓ Déployer avec succès sur Streamlit Cloud
\end{itemize}

\section{Points Forts}

\begin{enumerate}[leftmargin=*]
    \item \textbf{Architecture robuste}: Séparation claire des couches
    \item \textbf{Performance}: Génération en moins de 2 minutes
    \item \textbf{Qualité}: Tests complets, documentation exhaustive
    \item \textbf{Extensibilité}: Facile à étendre et maintenir
    \item \textbf{Utilisabilité}: Interface moderne et intuitive
\end{enumerate}

\section{Perspectives}

Le système constitue une base solide pour:

\begin{itemize}[leftmargin=*]
    \item Déploiement en production universitaire
    \item Extension à d'autres types de planification
    \item Intégration avec systèmes existants
    \item Développement d'applications mobiles
    \item Amélioration continue des algorithmes
\end{itemize}

\appendix

\chapter{Structure du Projet}

\begin{lstlisting}
DB PROJECT/
├── app.py                      # Application principale
├── requirements.txt            # Dépendances Python
├── database/
│   ├── schema.sql             # Schéma BD
│   ├── queries.sql            # Vues et fonctions
│   └── indexes.sql            # Indexes
├── src/
│   ├── database.py            # Module BD
│   ├── constraints.py         # Vérificateur
│   ├── fast_scheduler.py      # Algorithme optimisé
│   └── analytics.py           # Analytique
├── scripts/
│   ├── init_database.py       # Initialisation
│   ├── generate_data.py       # Génération données
│   └── test_all_functions.py # Tests
└── pages/
    ├── 1_Administration.py    # Interface admin
    ├── 2_Statistiques.py      # Analyses
    ├── 3_Départements.py      # Vues dept.
    └── 4_Consultation.py      # Plannings
\end{lstlisting}

\chapter{Références}

\begin{itemize}[leftmargin=*]
    \item PostgreSQL Documentation: \url{https://www.postgresql.org/docs/}
    \item Streamlit Documentation: \url{https://docs.streamlit.io/}
    \item Python psycopg2: \url{https://www.psycopg.org/docs/}
    \item Plotly Python: \url{https://plotly.com/python/}
\end{itemize}

\vfill
\begin{center}
\rule{0.8\textwidth}{0.4pt}

\textbf{Rapport Technique - Plateforme d'Optimisation des EDT}

Généré le \today

\rule{0.8\textwidth}{0.4pt}
\end{center}

\end{document}
